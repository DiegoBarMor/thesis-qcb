%%% detailed description of the protocols, software and algorithms used for the project
\chapter{Materials and methods} % at least 3 pages, maximum 10 pages; if more pages are needed, include an Appendix

[TODO: couple of sentences to introduce the sections]

%%%%%%%%%%%%%%%%%%%%%%%%%%%%%%%%%%%%%%%%%%%%%%%%%%%%%%%%%%%%%%%%%%%%%%%%%%%%%%%%
\section{Physical characterization}
  [TODO: brief introduction about the need of modelling the physical properties of the pocket into potential functions, to later generate 3D grids]

  \subsection{Stacking potential}
    [TODO: description of the model based in distance and alpha angles]
    [TODO: description of the data mining of PDB and statistical analysis of structures to get an idea of the shape and parameters of the model]

  \subsection{Hydrogen bonds potential}
    [TODO: description of a simple model based in a gaussian that takes only distance as parameter and assumes simultaneous positions where the ligand can have an ideal angle for the given distance]
    [TODO: description of how both acceptors and donors potentials are simplified by neglecting the actual position of the hydrogens, as their positions are many times flexible enough]

  \subsection{Electrostatic potential}
    [TODO: brief description of how APBS maps smoothly to the same concept of potential grid]
    [TODO: brief description of how APBS could need normalization (eg log10) because of the large range of values] --> but this was conceived
    \textcolor{red}{AFTER} seeing the results, should it still be here in "materials and methods" or in results?
    % \subsubsection{Normalizing the APBS grid}

  \subsection{Hydrophobicity potential}
    [TODO: brief description of how Alex's approach to Hydrophobicity was adatped for the UnityMol pipeline]


%%%%%%%%%%%%%%%%%%%%%%%%%%%%%%%%%%%%%%%%%%%%%%%%%%%%%%%%%%%%%%%%%%%%%%%%%%%%%%%%
\section{UnityMol}
  [TODO: brief technical details about implementing things in UnityMol (editing via the GUI and C\# scripting)]

  \subsection{Graphical User Interface}
    [TODO: describe how the user interacts with the pocket sphere mode of UnityMol via the new GUI submenu]

  \subsection{Potentials calculation}
    [TODO: describe how UnityMol communicates with an external Python script to calculate the potential grids]

  \subsection{Implementing the "cloud" representation}
    [TODO: describe the proposed method of visualizing the grids, by representing the grid as a "cloud" where color/opacity of the points is proportional to the relative grid values]


%%%%%%%%%%%%%%%%%%%%%%%%%%%%%%%%%%%%%%%%%%%%%%%%%%%%%%%%%%%%%%%%%%%%%%%%%%%%%%%%
