%%% critical analysis of the results obtained and frames them in the context of the international literature
\chapter{Discussion} % at least 5, maximum 10 pages

%%%%%%%%%%%%%%%%%%%%%%%%%%%%%%%%%%%%%%%%%%%%%%%%%%%%%%%%%%%%%%%%%%%%%%%%%%%%%%%%
\section{Characterization of Physicochemical Properties}
  \subsection{Stacking Potential}
    Multiple assumptions were imposed when modelling the stacking potential. Considering proximity as the only basis for what could correspond to stacking lead, as expeceted, to a lot of noise in the unfiltered interaction dataset. It is not surprising that focusing mostly in the nucleic-nucleic interactions yielded much cleaner statistics for parallel stacking, as this interaction is well known to be ubiquitous in nucleic acid systems \cite{rna_2015}.

    It would be expected that using only these statistics imposes a significant bias towards correctly predicting the stacking potential in RNA, while having arbitrary results with other systems. Indeed, all 5 relevant RNA benchmarks correctly predicted the stacking interactions.

    However, from the 7 relevant protein benchmarks, only two had mismatches between the potential and the ligand's aromatic groups (3DD0 and 6E9A), while the others had partial (1EHE and 5M9W) or complete overlaps (1BG0, 1H7L and 1IQJ). According to these observations, the stacking potential model seems to be generic for aromatic groups regardless of their structure, but is better suited for RNA systems.

    A possible improvement for the model would be to estimate the parameters for each aromatic residue, both for proteins and RNA systems. However, this could require a larger interaction dataset, or one built in a more refined manner, to yield significant improvements.

  \subsection{Hydrogen Bonds Potential}
    Despite the assumptions employed to simplify the configurational aspects of hydrogen bond interactions, this model is still very nonspecific on the heteroatoms it considers for calculating the hydrogen bond potentials. The model could be improved by calculating the potentials of heteroatoms that are probably not participating already in hydrogen bond interactions. This would require detecting base-pairing for RNA systems, or employing a general hydrogen bond detection algorithm for any molecular system.

    For the protein benchmarks relevant to hydrogen bonds, only 1H7L failed to display the expected hydrogen bond interactions from literature. The hydrogen bond acceptor potential was similarly accurate for the 7 pertinent RNA systems, with only 1AKX and 1I9V having slight issues in the prediction. However, for the hydrogen bond donor potential, only 4 of the 10 RNA systems had optimal results. The systems 2ESJ, 4F8U and 6TF3 predicted some of the interactions, while 1AKX, 1I9V and 7OAX (ligand SPM) failed to do so completely.

    Overall, the benchmarks show that the current model for the hydrogen bond potentials work correctly in many cases, but not always. Some improvements could be performed, for example optimizing the shape and parameters of the model via statistical methods, or by using less simplifications.

  \subsection{Electrostatic Potential}
    The protein benchmarks relevant for the electrostatic potentials (1BG0, 1EHE and 3EE4) succesfully reinforced what was expected from their respective literature sources. For other benchmarks, it was difficult discerning whether the visualization provides new insights or not, as many times the pocket was occupied by a single type of charge. This is particularly the case with RNA systems, where expectedly almost only negative charges are observed.

    The dependence on an external pipeline (APBS) can also hinder further changes on the visualization workflow, both in performance (additional computation time is required for every recalculation of the potential) and development flexibility (must always adhere to the kind of molecular systems APBS can handle). For example, some very large or particular molecular systems can halt the APBS calculation, which would in turn halt the visualization workflow. This issue is unique to the electrostatic potential, as there is full control over the other models, and any special cases would be handled accordingly.

  \subsection{Hydrophobicity Potential}
    The protein benchmarks where the hydrophobicity potential was relevant showed a great success, with only one system having suboptimal results (5M9W). Improvements can still be done for the model, for example to consider not only the residues but also the type of atom when calculating each probability distribution as hydrophobic or hydrophilic. Optimizing the shape and parameters of the model is also a relevant procedure that can be done.


%%%%%%%%%%%%%%%%%%%%%%%%%%%%%%%%%%%%%%%%%%%%%%%%%%%%%%%%%%%%%%%%%%%%%%%%%%%%%%%%
\section{Development of Visualization Methods}
  \subsection{General Considerations}
    In general, the volumetric potentials visualization methods proposed in this prototype provide interesting insights on the binding pockets of different target macromolecules. They can also provide additional value to the VR and AR visualization workflows being currently developed for UnityMol, once they are merged together in the future. However, the usefulness of volumetric representations can be limited when dealing with non interactive static media, such as images (as can be seen throughout this work), as collapsing the representations in a single 2D projection often results in convoluted visualization of the 3D data. This contrast is exemplified in the appendix website at \url{https://diegobarmor.github.io/thesis-qcb/}

    Anyhow, this methodology based on \textit{potential grids} provides a generic visualization technique, useful not only for the physical properties here presented, but also for new ones. Implementetation of additional volumetric properties is straightforward compatible, the only requirement being having a procedure for generating the three dimensional grid of points. Furthermore, the possibility of opening the output DX files in multiple visualization software provides great flexibility for end users.

    Additionally, the physical properties calculation and their visualization does not have to be bound to the binding pocket, and can potentially be applied to the \textit{whole volume} around a target macromolecule. Conversely, the methodology could also be used to focus on the physical properties of a given ligand, perhaps to perform an analysis inverse to the one here presented.

    The ligand could also be included to the current visualization pipeline to improve the trimming of the grid points. As a simple example, the volume already occupied by the ligand's atoms could be considered in the \textit{occupancy trimming} step. This can be specially useful if the ligand is changed interactively by the user, who could add, move or remove atoms and functional groups, and then recalculate the trimming mask for the grid to update the visualization accordingly.

    Another useful addition would be to implement the visualization of pharmacophores, and how they overlap with their respective potentials. This can be particularly useful if implemented for an interactive use, analogous to the concept described previously for ligands.

    Finally, it could be interesting to further explore the fieldlines representation (originally designed for the electrostatic potential), to consider its usage for visualizing statistical energy functions. For example, the potentials could be normalized and shifted to be in the range $[-0.5, 0.5]$, to simulate a signed potential and allow the fieldlines to display the gradient between low and high magnitudes of the potential.

  \subsection{Potentials Calculation}
    All three statistical energy functions here presented (stacking, hydrogen bonds and hydrophobicity potentials) rely on the same principle of summing up probability distributions centered in coordinate points of interest. However, the models differ on \textit{which} coordinates they employ and, more relevant for performance, \textit{how many} of them. The stacking potential needs to iterate only over the center of geometry of some residues, while the hydrogen bonds potential requires cycling through the positions of multiple heteroatoms. The longest cycles are often observed in the hydrophobicity potential, as it iterates over all the target molecule's atoms inside the PS.

    This is relevant when considering that each iteration involves computing a probability distribution over a large grid, which itself is not a trivial computation. The divisions, exponentiations and square roots involved renders the procedure computationally heavy, affecting performance at higher grid resolutions. At a resolution of $100 \times 100 \times 100$, the time required for computing each potential grid is in the order of a few decaseconds.

    For the visualization prototype here presented, the potential grids are calculated once per each static molecular system, so this performance is acceptable. However, if extending the visualization pipeline to dynamic systems is of interest (e.g. molecular dynamics \textbf{trajectories} or \textbf{interactive} manipulation of coordinates), computational performance becomes a relevant issue.

    A straightforward strategy for improving performance for dynamic systems is to recalculate the potentials only on some timesteps, e.g. on some specific frames (trajectories) or whenever the user decides to do so (interactive visualizations). However, the time required for each calculation would still too high for many practical uses.

    Another similarly lossy approach is to use lower resolutions for these cases. A resolution of $50 \times 50 \times 50$ is still informative (figure \ref{fig:results/resolution}) and is eigth times faster than the standard $100 \times 100 \times 100$, meaning a time scale of seconds per grid. The performance could also be easily improved by toggling off the RNDS trimming.

    Finally, a lossless strategy to improve performance would be to reutilize some of the calculations, perhaps taking advantage of the fact that the probability distributions centered around different points are numerically similar (or identical) and only differ in their position in the grid. Similarly, the distribution for a particular residue or atom could be calculated only if its coordinates change significantly, and so on.

  \subsection{Stacking Potential Visualization}
    The clarity of the visualization for the stacking potential was usually one of the best in the benchmarks. The fact that the probability distributions composing the grid were fewer and directional by nature meant that the representations tend to be less saturated by points, even before trimming (compare figures \ref{fig:results/visualize_stacking}.b and \ref{fig:results/visualize_hbonds}.c).

    After trimming, the stacking potential has the particularity that it is easier to infer many details about the possible interactions by just looking at these representations. Better defined tori (figures \ref{fig:benchmark/6tf3}.a and \ref{fig:appx_benchmark/2esj}.a) hint at broader regions of empty space, many times consisting of outer regions of the target, while highly trimmed and packed shapes (figures \ref{fig:benchmark/5bjo}.a and \ref{fig:appx_benchmark/3ee4}.a) imply more restrictive and specific pocket volumes.

    It is also easier to infer what would be the actual stacking interactions when combining the stacking potential and licorice representations of both the ligand and the relevant aromatic residues. This is shown throughout the benchmarks, being explicitely described for the RNA systems. A nice example of this is given in figure \ref{fig:benchmark/5bjo}.a.

    Furthermore, either cloud or isosurface representations are viable for this potential, as the general shapes of interest are maintained, and the density of points is usually low (figure \ref{fig:results/reprs_1}). However, the visualization can become convoluted in cases where many aromatic residues are of relevance, regardless of the representation method employed. Expectedly, this is a common occurance in RNA systems, as seen throughout the benchmarks, particularly cases such as figure \ref{fig:appx_benchmark/7oax1}.a.

  \subsection{Hydrogen Bonds Potential Visualization}
    In the case of hydroben bonds potentials, the clarity of the visualization was not always satisfactory. In some cases, a high density of heteroatoms in the binding pocket leads to a equally dense grid of potential points, which hinders practicality in examples such as figure \ref{fig:appx_benchmark/8eyv}.

    This is a consequence of the fact that each probability distribution generates grid points around all directions from any given atom of interest, at the same distance and potential value. Such homogeneity provides more information than what is many times useful (i.e. all possible positions for hydrogen bond interactions for any given ligand configuration), effectively affecting the practicality of the visualization.

    This issue also means that oftentimes there is a tradeoff between retaining the shapes of the potentials (figure \ref{fig:appx_benchmark/1ehe}.b) and actually being able to see through the volume (figures \ref{fig:appx_benchmark/1eby}.d and \ref{fig:appx_benchmark/5kx9}.c). In some cases, the isosurface representation can be tuned to maintain both pieces of information (figures \ref{fig:benchmark/1i9v} and \ref{fig:benchmark/2esj}), although this approach not always works well (figure \ref{fig:appx_benchmark/7oax1}.c). This issue is particularly relevant with dealing with static 2D images.

    The overall complexity of the visualization and the effort required to tune the representations imply that this potential should be improved at a model level, either by removing the amount of heteroatoms considered as interesting or by adjusting the underlying statistical energy function. Some ideas are already explored in the subsection \textit{Characterization of Physicochemical Properties / Hydrogen Bonds Potential} of this chapter.

  \subsection{Electrostatic Potential Visualization}
    The electrostatic potential is conceptually different from the other potentials, not only in modelling terms (physical energy function), but also in how the grid is composed. For the other cases, propensity values are only significant in specific regions near the atoms or residues of interest, but the electrostatic potential occupies the whole volume with its physical values. As so, this potential grids are very dense by nature and the representations need to address this fact.

    The benchmarks showed that for some cases, the electrostatic potential was homogeneous throughout the volume and was only informative in terms of what is the charge of the pocket (figure \ref{fig:appx_benchmark/1ofz}.d or any of the RNA systems). In other cases, the representations had to be tuned with the density of values in mind and some interesting behaviours could be observed (figures \ref{fig:benchmark/1ehe} and \ref{fig:appx_benchmark/1iqj}.d).

    Anyhow, a persistent issue with this visualization is that there is a tradeoff between the \textbf{clarity} of the representations and the ability to observe the \textbf{difference in magnitudes}. If the general shape of the charges in the volume is of interest, the representations previously mentioned work fine, but the information about which regions are more charged than others is lost.

    Conversely, if magnitudes are of interest, appropriate methods such as the cloud representation often suffer from clarity issues (figure \ref{fig:results/reprs_0}.a). Solving this problem requires the implementation of tresholds either during computation of the potentials (e.g. by trimming values that are out of a range of interest) or during visualization (e.g. setting of the isovalues for the isosurface representation). Both cases require the user to spend additional time tuning and experimenting with different tresholds, which could affect the practicality of the tool and the interpretation of the results.

    Another strategy could be to mix the static representations of the potential grid with a fieldlines representation, allowing the former to focus on \textit{clarity} and the latter on \textit{difference in magnitudes}. This idea is to be further explored in the future, as it is out of the scope of this prototype.

  \subsection{Hydrophobicity Potential Visualization}
    Despite the model similarities with the hydrogen bonds potentials, the visualization of the hydrophobicity potentials is oftentimes clear and informative, with few cases of convoluted representations found in the benchmarks (figure \ref{fig:appx_benchmark/6e9a}.e). This behaviour can be attributed to two main factors:

    \begin{itemize}
      \item The fact that the potential is signed allows for either focusing on the positive (hydrophobic) or negative (hydrophilic) grid points separately, as was done for figure \ref{fig:benchmark/1eby}, which is specially useful for complex intertwined potential grids. More importantly, when visualizing the totality of the potential grid, the additional information provided by the sign can be displayed by employing different color spaces or representation settings, as shown for the different benchmarks.

      \item Counterintuitively, by considering even more probability functions to fill the potential grid, the clarity of the representation improves significantly when compared to hydrogen bonds. This is due to the fact that directionality is no longer a relevant piece of information, as instead the hydrophobic potential emphasizes more in how the different signs compete for the same volume (similar to the electrostatic potential). Therefore, even if the shape of the potentials is sometimes neatly observed (figure \ref{fig:appx_benchmark/1ofz}.e), it is no longer relevant, lightening the responsability of the representation.
    \end{itemize}

    Following the previous observations, it can be said that the representations used for the hydrophobic potential are usually very flexible. Translucid settings are to be preferred, although the use of opaque isosurfaces can be useful to emphasize the contrast between the potential signs (figure \ref{fig:benchmark/5m9w}). In general, the visualization of this potential yielded satisfactory results.


%%%%%%%%%%%%%%%%%%%%%%%%%%%%%%%%%%%%%%%%%%%%%%%%%%%%%%%%%%%%%%%%%%%%%%%%%%%%%%%%
