%%% short summary, that provides the context and the objectives of the project, and summarizes the scientific problem, the techniques used and the results obtained
%%% In case of collaborative work, here is where the graduating student gives details on their contribution to the study
\chapter*{Abstract} % about 1 page in length
\addcontentsline{toc}{chapter}{Abstract}

The development of molecular visualization software plays an important role in many fields of molecular biology, such as the computer-aided drug discovery and design. Understanding the physical properties of binding pockets of target macromolecules, such as certain proteins and RNAs, is key in designing effective ligands for desired outcomes. Several interactions can occur between a binding pocket and its ligands, such as electrostatics, $\pi$-stacking, hydrogen bonds and hydrophobic effects, among others.

This prototype consisted of characterizing the physicochemical properties of the binding sites, developing visualization methods appropriate for the numerical results obtained, and benchmarking the whole pipeline with interesting protein and RNA systems. For the first step, mathematical models based in theoretical considerations and a statistical analysis (for the stacking interaction) were conceived. A visualization workflow was developed based in the framework provided by UnityMol, that allows users to define the volume of interest for a binding pocket, calculate their potentials and visualize the results. The volumetric data was mostly represented by isosurfaces, although a new visualization method was also proposed, denominated here \textit{cloud representation}.

The 10 protein and 10 RNA benchmarks provide multiple examples of how the prototype behaved under different circumstances. Results were varied, with the stacking potential having a great overall performance, followed by clear visualizations for the hydrophobic interactions. The electrostatic potential can still be polished in terms of visualization, while the hydrogen bonds potentials need to have their modelling and visualization aspects improved.

Altogether, the models and visualization workflow proposed in this prototype provide new insights into molecular systems of biological interest. It also serves as a proof of concept and a starting point for implementing further methodologies in a relatively unexplored molecular visualization niche, namely the physical characterization and representation of volumes of interest in molecular systems.
