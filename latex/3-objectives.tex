%%% summarize the project rationale, the scientific hypothesis and the key experiments that have been planned to test the hypothesis
\chapter{Objectives} % at least 1 page, not more than 2 pages

The workflow for this project consists of three main parts: \textit{characterization of physicochemical properties} of binding sites, \textit{development of visualization methods} appropriate for the numerical results of the former, and \textit{benchmarking} of the results. The \textbf{characterization of physicochemical properties} consists of devising or consolidating \textit{mathematical models} that approximate these properties in an accurate and intuitive way, parting from a PDB coordinates file for a protein or RNA system. The properties to be modelled are \textit{electrostatic interactions}, \textit{stacking}, \textit{hydrogen bonds} and \textit{hydrophobic effects}. For the development of these models, a \textit{quantitative and computational} approach is to be followed. Concepts from mathematics, physics, biochemistry and molecular biology are needed for the modelling aspect. Techniques from data mining and data analysis will also serve a purpose for the modelling of stacking interactions.

The \textbf{development of visualization methods} consists of adding new functionalities to the visualizer UnityMol, to allow a seamless interfacing with these models and an intuitive visualization of their results. A computational pipeline is to be developed, starting by user interactions in UnityMol, followed by an efficient calculation of the appropriate volumetric values, and finally the graphical representation of these numbers. A new molecular representation method for volumetric data is also proposed, denominated here \textbf{cloud representation}.

The hypothesis of this work is that the models describe the properties of the pocket volume in an accurate way, while the visualization of the data allows for an informed, ergonomic, fast and effective process for designing pertinent pharmacophores. The \textbf{benchmarking} stage will be the experimental approach for studying the hypothesis. Several structure files for crystallized protein-ligand and RNA-ligand complexes will be taken as empirical evidence of ideal ligand structure and spatial positioning. The physicochemical properties of the systems are to be calculated without the ligands present, then these values are to be visualized together with the ligand on top of them. If the pertinent functional groups overlap with the volumetric potentials predicted by the model, and the visualization methods allow for it to be an intuitive and seamless process, the hypothesis will be satisfied.
